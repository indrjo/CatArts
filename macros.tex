% !TEX program = lualatex
% !TEX root = ./CT.tex

% General notations
\newcommand\nil\varnothing
\newcommand\set[1]{\left\{#1\right\}}
\newcommand\angled[1]{\left\langle#1\right\rangle}
\renewcommand\bar\overline
\renewcommand\tilde\widetilde
\renewcommand\hat\widehat
\newcommand\tto\Rightarrow
\newcommand\nats{\mathbb N}
\newcommand\ints{\mathbb Z}
\newcommand\inout[1]{\textstyle#1\displaystyle}
\newcommand\epsln\varepsilon

% Notations pertaining Category Theory
\newcommand\cat\mathcal
\newcommand\obj[1]{\left\lvert#1\right\rvert}
\newcommand\functo[1]{\xrightarrow{\ #1\ }}
\newcommand\id{\mathtt 1}
\newcommand\comp{\operatorname{comp}}
\newcommand\inv[1]{#1^{-1}}
\newcommand\op[1]{#1^\text{op}}
%\newcommand\op[1]{\reflectbox{\(#1\)}}
\newcommand\opcat[1]{\op{\cat #1}}
%\newcommand\opcat[1]{\reflectbox{\(\cat #1\)}}
\newcommand\ev{\operatorname{ev}}

% Some categories
\newcommand\Set{\mathbf{Set}}
\newcommand\FinSet{\mathbf{FinSet}}
\newcommand\Grp{\mathbf{Grp}}
\newcommand\Top{\mathbf{Top}}
\newcommand\Mat{\mathbf{Mat}}
\newcommand\Vect{\mathbf{Vect}}
\newcommand\FDVect{\mathbf{FDVect}}
\newcommand\Dyn{\mathbf{Dyn}}
\newcommand\cn{\mathbf{Cn}}
\newcommand\cocn{\mathbf{CoCn}}

% Other shortcuts
\newcommand\naturaltr[5]{
  \begin{tikzcd}[ampersand replacement=\&]
    #2
      \ar["#4"{name=A}, r, bend left=35]
      \ar["#5"{name=B}, r, bend right=35, swap]
    \& #3
    \ar["#1", from=A, to=B, natural]
  \end{tikzcd}
}

\newcommand\adjoint[4]{
  \begin{tikzcd}[ampersand replacement=\&]
    #1 \ar["{#2}", r, shift left] \& #3 \ar["{#4}", l, shift left]
  \end{tikzcd}
}

% https://tex.stackexchange.com/a/602616
\DeclareFontFamily{U}{dmjhira}{}
\DeclareFontShape{U}{dmjhira}{m}{n}{ <-> dmjhira }{}
\DeclareRobustCommand{\yo}{\text{\usefont{U}{dmjhira}{m}{n}\symbol{"48}}}

%===
\def\comment#1{\color{red} #1 \color{black}}
\def\tba{\textbf{\color{blue} TBA}}
%===

\def\Set{\mathbf{Set}}
\def\Rel{\mathbf{Rel}}
\def\op{\text{op}}