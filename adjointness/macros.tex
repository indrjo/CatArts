% !TEX program = lualatex
% !TEX root = ./adjointness.tex

\newcommand\cat\mathcal
\newcommand\op[1]{#1^\text{op}}
\newcommand\Set{\mathbf{Set}}
\newcommand\FinSet{\mathbf{FinSet}}
\newcommand\Par{\mathbf{Par}}
\newcommand\Rel{\mathbf{Rel}}
\newcommand\Grp{\mathbf{Grp}}
\newcommand\Top{\mathbf{Top}}
\newcommand\Mat{\mathbf{Mat}}
\newcommand\obj[1]{\left\lvert#1\right\rvert}
\newcommand\inv[1]{#1^{-1}}
\renewcommand\bar\overline
\newcommand\functo[1]{\xrightarrow{\ #1\ }}
\newcommand\set[1]{\left\{#1\right\}}
\newcommand\id{\mathtt 1}
\newcommand\comp{\operatorname{comp}}

\newcommand\naturaltr[5]{
  \begin{tikzcd}[ampersand replacement=\&]
    #2
      \ar["#4"{name=A}, r, bend left=35]
      \ar["#5"{name=B}, r, bend right=35, swap]
    \& #3
    \ar["#1", from=A, to=B, natural] 
  \end{tikzcd}
}

\newcommand\adjoint[4]{%
  \begin{tikzcd}[ampersand replacement=\&]%
    #1 \ar["{#2}", r, shift left] \& #3 \ar["{#4}", l, shift left]%
  \end{tikzcd}%
}
