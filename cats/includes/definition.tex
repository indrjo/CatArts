% !TEX program = lualatex
% !TEX spellcheck = en_GB
% !TEX root = ../cats.tex

\section{Definition}

\begin{example}[Set Theory]
Here we have {\em sets} and {\em functions}. Whereas these of set and membership can be taken as primitive, the concept of function can be rigorously defined.
\begin{quotation}
For \(A\) and \(B\) sets, a function from \(A\) to \(B\) is a \(f \subseteq A \times B\) with this property: for every \(x \in A\) there exists one and only one \(y \in B\) such that \((x, y ) \in f\).
\end{quotation}
We write \q{\(f : A \to B\)} to mean \q{\(f\) is a function from \(A\) to \(B\)} and prefer \q{\(f(x) = y\)} instead of \q{\((x, y) \in f\)}. As things are set now, we can compose two functions: for \(A\), \(B\) and \(C\) sets and \q{consecutive} functions
\[A \functo{f} B \functo {g} C ,\]
the {\em composite} of \(g\) and \(f\) is the function
\begin{align*}
g \circ f &: A \to C \\
g \circ f(x) &:= g\big(f(x)\big) .
\end{align*}
Such operation has some remarkable properties.
\begin{enumerate}
\item For every set \(A\) the function \(\id_A : A \to A\) defined by \(\id_A(x) = x\) is such that for every set \(B\) and function \(g : B \to A\) we have
\[\id_A \circ g = g\]
and for every set \(C\) and function \(h : A \to C\) we have
\[h \circ \id_A = h .\]
Here, \(\id_A\) is the {\em identity} for \(A\).
\item \(\circ\) is associative, that is for \(A\), \(B\), \(C\) and \(D\) sets and
\[A \functo{f} B \functo{g} C \functo{h} D\]
functions, we have the identity
\[(h \circ g) \circ f = h \circ (g \circ f) .\]
\end{enumerate}
\end{example}

\begin{example}[Topology]
A {\em topological space} is a set where some of its subsets have the status of \q{open}. And a {\em continuous function} is a function that cares about such label: precisely, for \(X\) and \(Y\) topological spaces, a function \(f : X \to Y\) is said continuous whenever for every open set \(U \subset Y\) also \(\inv f U \subset X\) is open. It is not hard to show that composing two continuous function yields another continuous function. Since a continuous function is a function and the identity is continuous, the properties listed in the previous example are naturally inherited: you only need to replace the word \q{set} with \q{topological space} and \q{function} with \q{continuous function} and things work.
\end{example}

\begin{exercise}[Topology, again]
This is a little variation of the previous example. Take topological spaces and {\em open functions}: now, can you stand out some pattern, like that of the previous two examples? Here, for \(X\) and \(Y\) topological spaces, a function \(f : X \to Y\) is said open whenever for every open set \(U \subseteq X\) the set \(f U \subseteq Y\) is open too. 
\end{exercise}

\begin{exercise}
One can provide a generous amount of examples of contexts presenting that leitmotif when when things are built upon Set Theory. After all, groups are sets with some additional ingredients, homomorphisms are functions that cares about the group structure, composing two such functions lead to a homomorphism; and composition comply the same laws highlighted in the previous two examples. The same applies to vectorial spaces, measure spaces, probability spaces, \dots{} Make some examples by yourself.
\end{exercise}

\begin{definition}[Categories]
A {\em category} amounts at assigning some things called {\em objects} and, for each couple of objects \(a\) and \(b\), of some other things named {\em morphisms} from \(a\) to \(b\). We write \(f : a \to b\) to say that \(f\) is a morphism from \(a\) to \(b\), where \(a\) is the {\em domain} of \(f\) and \(b\) the codomain. Besides, for \(a\), \(b\) and \(c\) objects and \(f : a \to b\) and \(g : b \to c\) morphisms, there is associated the {\em composite morphism}
\[gf : a \to c .\]
All those things are regulated by the following axioms:
\begin{enumerate}
\item for every object \(x\) there is a morphism, \(\id_x\), from \(x\) to \(x\) such that for every object \(y\) and morphism \(g : y \to x\) we have
\[\id_x g = g\]
and for every object \(z\) and morphism \(h : x \to z\) we have
\[h \id_x = h ;\]
\item for \(a\), \(b\), \(c\) and \(d\) objects and morphisms
\[a \functo{f} b \functo{g} c \functo{h} d\]
we have the identity
\[(h g) f = h (g f) .\]
\end{enumerate}
\end{definition}

The above definition has the aim to generalize the \q{objects-morphisms-compositionality} pattern to a broader class of situations. In fact, it is important to always keep in mind that objects need not to be sets (with a structure, possibly) and morphisms need not to be functions (with some property, eventually) and composition need not to be a composition of functions.

\begin{figure}
\centering
\begin{tikzpicture}
\node [circle] (obj) {\(\bullet\)};
\foreach \i in {0, ..., 4} {
	\draw [-{To[length=3.5pt, width=4pt]}] (obj)
		.. controls (5+72*\i-15:1cm)
			and (5+72*\i+15:1cm) .. (obj);
	\node [anchor=center] at (5+72*\i:1.1cm) {\(f_\i\)};
}
\end{tikzpicture}
\caption{A monoid with five elements pictured as a category}
\label{fig:MonsAreCats}
\end{figure}

\begin{example}[Monoids]
Consider a category \(\cat G\) with a single object, that we indicate with a bare \(\bullet\) --- see figure~\ref{fig:MonsAreCats}. All of its morphisms have \(\bullet\) as domain and codomain: this fact implies the composite of two morphisms \(\bullet \to \bullet\) is a morphism \(\bullet \to \bullet\) too. This motivates us to proceed as follows: let \(G\) be the collection of the morphisms of \(\cat G\) and consider the function
\[G \times G \to G\,,\ (x, y) \to xy ,\]
that is the operation of composing morphisms. Being \(\cat G\) a category implies this function is associative and \(\cat G\) has the identity of \(\bullet\), that is \(G\) has one element we call \(1\) and such that \(f1 = 1f = f\) for every \(f \in G\). In other words, we are saying \(G\) is a monoid.\newline
Conversely, take a monoid \(G\) and any thing you want: make such thing acquire the status of object and the elements of \(G\) that of morphisms; in that case, the operation of \(G\) has the right to be called composition because the axioms of monoid say so.\newline
The conclusion is: a monoid \q{is} a category with a single object. \(\bullet\) is something we cared of only because by definition morphisms need objects and it has no role other than this.
\end{example}

In Mathematics, a lot of things are monoids, so this is nice.

\begin{example}[Preordered sets, rephrased]
In this example we show how from a preordered sets we can have a category. A {\em preordered set} (sometimes contracted as {\em proset}) consists of a set \(A\) and a relation \(\le\) on \(A\) such that:
\begin{enumerate}
\item \(x \le x\) for every \(x \in A\);
\item for every \(x, y, z \in A\) we have that if \(x \le y\) and \(y \le z\) then \(x \le z\).
\end{enumerate}
Now we do this: for \(x, y \in A\), whenever \(x \le y\) take \((a, b) \in A \times A\). We operate with these couples as follows:
\begin{equation}
(y, z) (x, y) := (x, z), \label{eqn:ProsetCirc}
\end{equation}
where \(x, y, z \in A\). This definition is perfectly motivated by (2): in fact, if \(x \le y\) and \(y \le z\) then \(x \le z\), and so there is \((x, z)\). By (1), for every \(x \in A\) we have the couple \((x, x)\), which has the following property: for every \(y \in A\)
\begin{equation}
\begin{aligned}
& (x, y) (x, x) = (x, y) & \text{for every } y \in A \\
& (x, x) (z, x) = (z, x) & \text{for every } z \in A .
\end{aligned}\label{eqn:ProsetId}
\end{equation}
Another remarkable feature is that for every \(x_1, x_2, x_3, x_4 \in A\)
\begin{equation}
\big((x_3, x_4)(x_2, x_3)\big)(x_1, x_2) = (x_3, x_4)\big((x_2, x_3)(x_1, x_2)\big)\label{eqn:ProsetAssoc}
\end{equation}
We have a category indeed: its objects are the elements of \(A\), the morphisms are the couples \((x, y)\) such that \(x \le y\) and~\eqref{eqn:ProsetCirc} gives the notion of composition; \eqref{eqn:ProsetId} says what are identities while~\eqref{eqn:ProsetAssoc} tells compositions are associative.
\end{example}

A lot of things are prosets, so this is nice.

\begin{example}[Matrices]
We need to clarify some terms and notations before. Fixed some field \(k\), for \(m\) and \(n\) positive integers, a {\em matrix} of type \(m \times n\) is a table of elements of \(k\) arranged in \(m\) rows and \(n\) columns:
\[\begin{pmatrix}
x_{1,1} & x_{1,2} & \cdots{} & x_{1,n} \\
x_{2,1} & x_{2,2} & \cdots{} & x_{2,n} \\
\vdots  & \vdots  &          & \vdots  \\
x_{m,1} & x_{m,2} & \cdots{} & x_{m,n}
\end{pmatrix}\]
If \(A\) is the name of a matrix, then \(A_{i, j}\) is the element on the intersection of the \(i\)th row and the \(j\)th column. Matrices can be multiplied: if \(A\) and \(B\) are matrices of type \(m \times n\) and \(n \times r\) respectively, then \(AB\) is the matrix of type \(m \times r\) where
\[(AB)_{i, j} := \sum_{s = 1}^n A_{i, s} B_{s ,j} .\]
Our experiment is this: consider the positive integers in the role of objects and, for \(m\) and \(n\) integers, the matrices of type \(m \times n\) as morphisms from \(n\) to \(m\); now, take \(AB\) as the composition of \(A\) and \(B\). Let us investigate whether categorial axioms hold.
\begin{itemize}
\item For \(n\) positive integer, we have the {\em identity matrix} \(I_n\), the one of type \(n \times n\) defined by
\[(I_n)_{i, j} = \begin{cases}
1 & \text{if } i = j \\
0 & \text{otherwise}
\end{cases}\]
One, in fact, can verify that such matrix is an \q{identity} in categorial sense: for every positive integer \(m\), an object, and for every matrix \(A\) of type \(m \times n\), a morphism from \(n\) to \(m\), we have
\[A I_n = A ,\]
that is composing \(A\) with \(I_n\) returns \(A\); similarly, for every positive integer \(r\) and for every matrix \(B\) of type \(r \times n\) we have
\[I_n B = B .\]
\item For \(A\), \(B\) and \(C\) matrices of type \(m \times n\), \(n \times r\) and \(r \times s\) respectively, we have
\[(AB)C = A(BC) .\]
Again, this identity can be regarded under a categorial light.
\end{itemize}
The category of matrices on a field \(k\) just depicted is written \(\Mat_k\).
\end{example}

\begin{example}[Chain complexes]
[Yet to write\dots{}]
\end{example}
